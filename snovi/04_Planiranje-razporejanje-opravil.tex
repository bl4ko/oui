\section{Transportna plast}

\subsection{UDP}
\textbf{Kontrolna vsota} (Internet Checksum): 
\setlist{nolistsep}
\begin{itemize}[leftmargin=*,noitemsep]
    \item \blue{posiljatelj} sesteje 16 bitne besede in shrani \textbf{eniski komplement = kontrolna vsota}
    \item \blue{prejemnik} sesteje 16 bitne besede skupaj s kontrolno vsoto -> dobiti mora same enice
\end{itemize}



\subsection{TCP}

\subsubsection{Vrste protokolov}
\color{green} Posredno potrjevanje \color{black}: uporaba samo ACK\\
\color{green} Neposredno potrjevanje \color{black}: uporaba ACK in NAK\\
\color{green} Sprotno potrjevanje \color{black} (stop\&wait protocol), mora pred oddajo naslednjega pakta cakati na potrditev prejsnega.\\
\color{green} Tekoce posiljanje \color{black} (pipelined) istocasno potuje vec paketov brez sprotnega cakanja na njihovo potrditev. Poznamo dve vrrsti:
\green{Ponavljanje \textbf{N nepotrjenih}} (go back N): posiljatelj hrani "okno" najvec dovoljenih nepotrjenih paketov (tak protokol imenujemo tudi \textbf{protokol z drsecim oknom}), ACK(n)
\includegraphics[width=6cm]{ponavljanje-N-nepotrjenih.png}\\
\green{Ponavljanje \textbf{izbranih}} (selective repeat): prejemnik potrjuje vsak prejeti paket posamezno, jih shranjuje v napacnem vrstenm redu ter jih \textbf{sortira} pred dostavo aplikaciji. Posiljatelj hrani \textbf{stoparico za vsak posamezni}
\includegraphics[width=7cm]{ponavljanje-izbranih.png}

\subsubsection{TCP prejemnik}
\includegraphics[width=7cm]{tcp.png}


\subsubsection{TCP segment}
\includegraphics[width=7cm]{TCP-segment.png}


\subsubsection{ocenjevanje RTT (round trip time)}
$\text{OcenjeniRTT}[1] = \text{IzmerjeniRTT}[1]$\\
$\green{\text{OcenjeniRTT[i]}} = (1-\alpha) \cdot \text{OcenjeniRTT[i-1]} + \alpha \cdot \text{IzmerjeniRTT[i]}$\\
$\red{\text{DevRTT[i]}}=(1-\beta)\cdot \text{DRTT[i-1]} + \beta \cdot \mid \text{IRTT}[i]-\text{ORTT}[i]\mid$\\
$\blue{\text{CakalniInterval[i]]}}=\green{\text{OcenjeniRTT[i]}}+ 4\cdot \red{\text{DevRTT[i]}}$



\subsubsection{ThreeWay handshake}
\blue{source}: SYN=1 (random generated seq. number)\\
\blue{dest}: SYN=1,ACK=1 (potrjujem to sekvencno stevilko, mimogrede moja je...)\\
\blue{source}: ACK=1 (potrjujem tvojo sekvencno stevilko)

\subsubsection{Congestion Control}
\green{Najvecji mozen prenos}: $\min(cwnd, rwnd)$\\
\green{cwnd}: velikost zamastvenega okna v MSS (koliko podatkov lahko posljem brez ACK)\\
\green{ssthresh}: meja ki doloca prehod med fazama \textbf{slow start} in \textbf{congestion avoidance}\\
\green{Slow start}
\begin{itemize}[leftmargin=*, noitemsep]
    \item Ko: \blue{cwnd < ssthresh}
    \item Uspesen prenos skupine paketov: $cwnd = cwnd * 2$
    \item oziroma $\text{cwnd}=\text{cwnd}+1$ (za vsak ACK)
    \item \blue{Izguba - 3x ACK}\begin{itemize}[leftmargin=*]
        \item Tahoe: ssthresh=cwnd/2, cwnd=1    
        \item Reno: ssthresh=cwnd/2, cwnd = ssthresh + 3 (Fast Recovery)
    \end{itemize}
    \item \blue{Timeout}:\begin{itemize}[leftmargin=*]
        \item Tahoe/Reno: ssthresh=cwnd/2, cwnd=1
    \end{itemize}
\end{itemize}
\green{Congestion avoidance}
\begin{itemize}[leftmargin=*, noitemsep]
    \item Ko: \blue{cwnd $\geq$ ssthresh}
    \item Uspesen prenos skupine paketov (cwnd): $cwnd = cwnd + 1$
    \item $\text{cwnd}=\text{cwnd} + 1/\text{cwnd}$ za vsak ACK
    \item \blue{Izguba - 3x ACK}\begin{itemize}[leftmargin=*]
        \item Tahoe: $ssthresh=cwnd/2$, $cwnd=1$
        \item Reno: $ssthresh=cwnd/2$, $cwnd = ssthresh + 3$ (Fast Recovery)
    \end{itemize}
    \item \blue{Timeout}:\begin{itemize}[leftmargin=*]
        \item Tahoe/Reno: $ssthresh=cwnd/2$, $cwnd=1$
    \end{itemize}
\end{itemize}




